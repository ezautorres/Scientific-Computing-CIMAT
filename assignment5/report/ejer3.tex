% -----------------------------------------------------------------------------------------
\vspace{5mm}
{\color{lightgray} \hrule}
\begin{enumerate} \setcounter{enumi}{2}
	\item ¿Cuál es el algoritmo que usa scipy.stats.uniform para generar números aleatorios? ¿Cómo se pone la semilla? ¿y en R? [1 punto]
\end{enumerate}

\textcolor{BrickRed}{\it Respuesta:}

\begin{itemize}

\item En Python, la función scipy.stats.uniform utiliza el generador de números pseudoaleatorios de NumPy (numpy.random.Generator) para generar números aleatorios a partir de la distribución uniforme. El algoritmo detrás de este generador se basa en la implementación del método PCG64 (Permuted Congruential Generator) por defecto en versiones recientes de NumPy, que es rápido, eficiente y cumple con buenos estándares de aleatoriedad. La semilla se fija con numpy.random.seed()

PCG64 (Permuted Congruential Generator 64) es un algoritmo de generación de números pseudoaleatorios de alta calidad que fue diseñado para ofrecer mejor rendimiento y mayor seguridad frente a defectos comunes en generadores más simples. PCG64 utiliza un generador congruencial lineal (LCG), que es uno de los métodos más antiguos y populares para generar números pseudoaleatorios (como el implementado en el \textcolor{red}{ejercicio 2}). El LCG genera secuencias de números pseudoaleatorios utilizando la fórmula:
$$
X_{n+1} = (a \cdot X_n + c) \mod m
$$
Donde $X_n$ es el estado interno del generador, $a$, $c$ y $m$ son constantes. En el caso de PCG, $m$ es una potencia de $2$, lo que permite una implementación muy eficiente. PCG64 trabaja con enteros de 64 bits, lo que le permite tener un período extremadamente largo, lo que significa que puede generar una cantidad gigante de números aleatorios antes de repetir una secuencia. En particular, el período es $2^{128}$, lo que garantiza que las secuencias sean muy largas antes de repetir.

\item En R, la función equivalente es runif(), que genera números aleatorios de una distribución uniforme. R utiliza el generador de números pseudoaleatorios Mersenne Twister por defecto. La semilla se fija con set.seed().

El Mersenne Twister tiene un período de $2^{19937}-1$ (más largo que el de Numpy y Scipy, aunque ambos pasan las famosas pruebas de aleatoriedad), lo que significa que puede generar una cantidad inmensa de números aleatorios antes de repetir la secuencia. El Mersenne Twister es un generador de números pseudoaleatorios basado en operaciones de desplazamiento de bits y mezclas no lineales para generar números de alta calidad.

El algoritmo Mersenne Twister en $R$ genera números pseudoaleatorios utilizando un estado interno de $624$ enteros de $32$ bits. Mezcla y transforma estos valores con operaciones de desplazamiento y $XOR$ para producir números de alta calidad, con un largo período de $2^{19937}-1$.
\end{itemize}