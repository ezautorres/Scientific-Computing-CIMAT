% -----------------------------------------------------------------------------------------
\vspace{5mm}
{\color{lightgray} \hrule}
\begin{enumerate} \setcounter{enumi}{3}
	\item ¿En scipy que funciones hay para simular una variable aleatoria genérica discreta? ¿tienen preproceso? [1 punto]
\end{enumerate}

\textcolor{BrickRed}{\it Respuesta:}

En SciPy, para simular una variable aleatoria discreta genérica, se puede utilizar principalmente dos funciones dentro del módulo scipy.stats: \url{https://docs.scipy.org/doc/scipy/reference/generated/scipy.stats.rv\_discrete.html}

\begin{itemize}
	\item \textit{rv\_discrete:}
\end{itemize}	
	Es una clase para definir una distribución discreta genérica. Se pueden especificar los posibles valores de la variable aleatoria y sus probabilidades asociadas.
\begin{itemize}	
	\item \textit{rv\_sample (en scipy.stats.qmc)::}
\end{itemize}
Esta es otra función útil para variables aleatorias discretas genéricas, aunque se encuentra en el submódulo qmc para \textit{muestreo cuasi-Monte Carlo}. Permite definir una distribución discreta en función de los valores y probabilidades que asignas.

Para la parte de preprocesamiento, en la documentación oficial de Scipy se menciona que antes de generar las muestras, ambas funciones verifican que:
\begin{itemize}
	\item La suma de las probabilidades sea $1$.
	\item Los valores estén correctamente definidos.
\end{itemize}

Este es el preprocesamiento que se realiza para garantizar que los datos sean válidos antes de la simulación. Además, en \textit{rv\_discrete}, las probabilidades se almacenan internamente de manera eficiente, permitiendo un acceso rápido para el muestreo durante la simulación.