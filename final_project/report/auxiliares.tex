% ------------------------------------------------------------------------------------
\newpage
\section{Funciones Auxiliares} \label{sec:funauc}

En esta parte, se describen las funciones auxiliares implementadas en el archivo:
\begin{center}
	\textcolor{mediumblue}{funciones\_auxiliares.py}:
\end{center}

\textbf{Para el burn\_in y las modas}
\begin{itemize}
	\item \textit{moving\_average()}
\end{itemize}
Esta función calcula la media móvil de un arreglo unidimensional, una herramienta común para suavizar series temporales o datos ruidosos. Toma como entrada un arreglo y un tamaño de ventana que determina el número de elementos considerados para cada promedio. Utiliza convolución para generar un nuevo arreglo donde cada elemento es el promedio de los valores en la ventana correspondiente, permitiendo una mejor visualización de las tendencias subyacentes en los datos. Esta función se utiliza dentro de la función siguiente:

\begin{itemize}
	\item \textit{estim\_burn\_in\_and\_modes()}
\end{itemize}
Esta función tiene dos propósitos principales: estimar el burn-in de una cadena de Markov y calcular las modas de los parámetros. Para el burn-in, utiliza la estabilización de la media móvil, identificando el punto donde la variación de esta se vuelve insignificante. Esto ayuda a descartar las iteraciones iniciales que aún no han alcanzado el régimen estacionario. Además, estima las modas de los parámetros analizando los histogramas de las distribuciones y seleccionando los valores más probables.

\textbf{Graficación}
\begin{itemize}
	\item \textit{marginal\_evolution\_burn\_in()}
\end{itemize}
Esta función visualiza la evolución de las cadenas marginales para cada parámetro de una cadena de Markov, proporcionando información sobre su convergencia. Genera gráficos donde se destacan los valores iniciales de cada cadena y el punto estimado de burn-in con líneas verticales, facilitando la identificación de regiones estables. Es especialmente útil para evaluar el comportamiento de los parámetros a lo largo del tiempo y confirmar visualmente si la cadena ha alcanzado el equilibrio.

\begin{itemize}
	\item \textit{histograms\_chain()}
\end{itemize}
Genera histogramas de las distribuciones posteriores de los parámetros después de descartar el burn-in. Estos histogramas permiten observar la densidad de probabilidad de cada parámetro, resaltando las modas con líneas verticales. Adicionalmente, para el ejercicio 1, se incluyen los datos y esta función incluye un histograma de los mismos y superpone una distribución binomial estimada para comparar la información posterior con los datos reales.

\begin{itemize}
	\item \textit{trayectoria2d3d()}
\end{itemize}
Permite analizar visualmente cómo una cadena de Markov explora el espacio de parámetros. En el caso de dos parámetros, genera un gráfico bidimensional donde se observa la trayectoria de la cadena, marcando tanto el punto inicial como la moda estimada. Si hay tres parámetros, añade una visualización tridimensional y gráficos proyectados en los planos bidimensionales correspondientes. Esta función es especialmente valiosa para comprender cómo se comporta la cadena en espacios paramétricos de mayor dimensión, ayudando a identificar patrones, convergencia y regiones de alta probabilidad.