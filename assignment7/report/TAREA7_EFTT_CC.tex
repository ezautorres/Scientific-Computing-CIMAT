%%%%%%%%%%%%%%%%%%%%%%%%%%%%%%%%%%%%%%%%%
% 1. Document Class

\documentclass[a4paper,12pt]{report}
\renewcommand{\familydefault}{\sfdefault} % tipo de letra general

%%%%%%%%%%%%%%%%%%%%%%%%%%%%%%%%%%%%%%%%%
% 2. Paquetes
\usepackage[utf8]{inputenc}
\usepackage[english]{babel}
\usepackage[top = 2.5cm, bottom = 2.5cm, left = 2.5cm, right = 2.5cm]{geometry}
\usepackage{ae,aecompl,amsmath,amsfonts,amssymb,graphicx}
\usepackage{wrapfig,float,titlesec,titletoc,fancyhdr,parskip}
\usepackage{mathrsfs,subfig,color,multicol,amscd,eso-pic}
\usepackage[usenames,dvipsnames,svgnames,table]{xcolor}
\usepackage{bbm}
\usepackage[most]{tcolorbox}
\usepackage{subcaption}
\usepackage[T1]{fontenc}
\usepackage[colorlinks=true,linkcolor={RedOrange},citecolor={Purple},breaklinks=true,linktocpage]{hyperref}
%\spanishdecimal{.}
\usepackage{enumitem}
\usepackage{listings,transparent}
\lstset{language=Matlab}
\definecolor{codegreen}{rgb}{0,0.6,0}
\definecolor{codegray}{rgb}{0.5,0.5,0.5}
\definecolor{codepurple}{rgb}{0.44,0.0,0.1}
\definecolor{mediumblue}{rgb}{0.0,0.0,0.8}
%\definecolor{electricviolet}{rgb}{}
\definecolor{backcolour}{rgb}{0.95,0.95,0.92}
\lstdefinestyle{mystyle}{
	backgroundcolor=\color{backcolour},   
	commentstyle=\color{codegreen},
	keywordstyle=\color{blue},
	numberstyle=\tiny\color{codegray},
	stringstyle=\color{codepurple},
	basicstyle=\ttfamily\footnotesize,
	breakatwhitespace=false,         
	breaklines=true,                 
	captionpos=t,                    
	keepspaces=true,                 
	numbers=left,                    
	numbersep=6pt,                  
	showspaces=false,                
	showstringspaces=false,
	showtabs=false,                  
	tabsize=2 }
\lstset{style=mystyle}
\renewcommand\lstlistingname{\bf Código}
\renewcommand\lstlistlistingname{Códigos}

%%%%%%%%%%%%%%%%%%%%%%%%%%%%%%%%%%%%%%%%%
% Marca de agua
\AddToShipoutPicture{
	\put(0,0){
		\parbox[b][\paperheight]{\paperwidth}{
			\vfill
			\centering
			{\transparent{0.2}\includegraphics[scale=0.8]{IMAGENES/logo}}
			\vfill
		}
	}
}

%%%%%%%%%%%%%%%%%%%%%%%%%%%%%%%%%%%%%%%%%
% Estilo de paginas
\pagestyle{fancy}
\fancyhf{}
\lhead{\footnotesize \color{BrickRed} \large Cómputo Científico} % esquina izquierda
\rhead{\footnotesize \color{BrickRed} \large Tarea 7} % esquina derecha
\cfoot{\color{NavyBlue} \bfseries\thepage }

\title{Tarea 6}
\author{Ezau Torres}

%%%%%%%%%%%%%%%%%%%%%%%%%%%%%%%%%%%%%%%%%%%%%%
% DOCUMENTO

\begin{document}
	\begin{minipage}{0.15\textwidth}
		\centering
		\includegraphics[width=\textwidth]{IMAGENES/logo}
	\end{minipage}
	\begin{minipage}{0.85\textwidth}
		\centering
		\textbf{\large Cómputo Científico para Probabilidad, Estadística y Ciencia de Datos}\\
		\medskip
		\text{\large Ezau Faridh Torres Torres} \\
		\medskip
		\textbf{\large TAREA 7: MCMC: Metropolis-Hastings II}\\
		\medskip
		\textbf{Fecha de entrega:} 23/Oct/2024.
	\end{minipage}

\vspace{5mm}

%%%%%%%%%%%%%%%%%%%%%%%%%%%%%%%%%%%%%%%%%
% CONTENIDO
%%%%%%%%%%%%%%%%%%%%%%%%%%%%%%%%%%%%%%%%%

\textcolor{BrickRed}{\bf NOTA:}  Los ejercicios se encuentran repartidos en los archivos:
\begin{itemize}
	\item \textcolor{mediumblue}{ejercicio1\_tarea7.py}
	\item \textcolor{mediumblue}{ejercicio2\_tarea7.py}
	\item \textcolor{mediumblue}{ejercicio3\_tarea7.py}
\end{itemize}

\textbf{Con el algoritmo Metropolis-Hastings (MH), simular lo siguiente:}

% -----------------------------------------------------------------------------------------
\vspace{5mm}
{\color{lightgray} \hrule}
\begin{enumerate}
	\item a) Definir la cdf inversa generalizada $F_{X}^{-}$ y demostrar que en el caso de variables aleatorias continuas esta coincide con la inversa usual.
	
	b) Demostrar además que en general para simular de $X$ podemos simular $u\sim U(0,1)$ y $F_{X}^{-} (u)$ se distribuye como $X$. [1 punto]
\end{enumerate}

\textcolor{BrickRed}{\it Respuesta:}

{\color{blue}
	\begin{enumerate}[label=\alph*), start=1]
		\item  
\end{enumerate}}
Sea $X$ una variable aleatoria con función de distribución acumulada (CDF) $F_{X}(x) =  \mathbb{P} \left(X\leq x\right)$, la \textbf{CDF inversa generalizada} $F_{X}^{-} (q)$ se define como:
\begin{equation}
	F_{X}^{-} (q) = \inf\left\{ x\in\mathbb{R} : F_{X}(x) \geq q \right\}
\end{equation}
para $q \in (0,1)$.

Entonces, si $X$ es una variable aleatoria continua, su CDF $F_{X}(x)$ es una función no decreciente y continua en todo su dominio aunque $F_{X}(x)$ puede ser constante en ciertos intervalos (donde la densidad de probabilidad $f_{X}(x)=0$). Entonces, se tienen dos casos:
\begin{itemize}
	\item Cuando $F_{X}(x)$ es estrictamente creciente, para cada $q\in(0,1)$, existe un único $x$ tal que $F_{X}(x)=q$ gracias a que la función es biyectiva (al ser continua y estrictamente creciente). Este $x$ es el único valor que satisface $F_{X}(x)=q$ y por lo tanto, la inversa usual existe, es única y cumple que $F_{X}^{-1}(q)=x$.
	
	Además, por la definición de la inversa generalizada, se busca el ínfimo de los valores de $x$ tales que $F_{X}(x)\geq q$. Como $F_{X}(x)$ es estrictamente creciente, este conjunto solo contiene un valor de $x$, que es el que satisface $F_{X}(x)=u$. Esto implica que:
	\begin{equation}
		F_{X}^{-}(q) = \inf\left\{ x\in\mathbb{R} : F_{X}(x) \geq q \right\} = F_{X}^{-1}(q).
	\end{equation}
	Por lo tanto, cuando $F_{X}(x)$ es estrictamente creciente, la inversa generalizada $F_{X}^{-}(q)$ coincide exactamente con la inversa usual $F_{X}^{-1}(q)$  ya que en ambos casos se obtiene el único valor $x$ que satisface $F_{X}(x) = q$.
	
	\item En donde $F_{X}(x)$ es constante, la inversa generalizada sigue siendo válida. La definición de $F_{X}^{-}(q)$ como el ínfimo de los $x$ tales que $F_{X}(x)\geq q$ asegura que siempre obtenemos el valor correcto de $x$, incluso si la CDF no es estrictamente creciente.
\end{itemize}

\newpage
{\color{blue}
	\begin{enumerate}[label=\alph*), start=2]
		\item  
\end{enumerate}}
Sea $u\sim U(0,1)$, queremos probar que $F_{X}^{-}(u)$ se distribuye como $X$, es decir, que
\begin{equation}
	\mathbb{P} \left(F_{X}^{-}(u) \leq x\right) = \mathbb{P}\left(X\leq x\right) = F_{X}(x).
\end{equation}

Por definición de la inversa generalizada, se tiene que $F_{X}^{-}(u) \leq x$ implica que el valor de $u$ debe estar en el rango de valores menores o iguales a  $F_{X}(x)$, es decir, $u\leq F_{X}(x)$. Con esto,
\begin{equation} \label{eq:4}
	\mathbb{P} \left(F_{X}^{-}(u) \leq x\right) = \mathbb{P}\left(u\leq F_{X}(x)\right)
\end{equation}

Finalmente, dado que $u\sim U(0,1)$, se tiene que la probabilidad de que $u\leq a\in[0,1]$ es exactamente $a$. Entonces, $\mathbb{P}\left(u\leq F_{X}(x)\right) = F_{X}(x)$. Sustituyendo esto en \eqref{eq:4}, se tiene que $F_{X}^{-}(u)\sim X$.
% ------------------------------------------------------------------------------------
\newpage
{\color{lightgray} \hrule}
\begin{enumerate} \setcounter{enumi}{1}
	\item Considere los tiempos de falla $t_1, \dots, t_n$ con distribución $Weibull(\alpha, \lambda)$:
	\begin{equation} \label{eq:7}
		f(t_i | \alpha, \lambda) = \alpha \lambda t_{i} ^{\alpha-1} e^{-t_{i}^{\alpha}\lambda}
	\end{equation}
	Se asumen como a priori $\alpha \sim exp(c)$ y $\lambda|\alpha \sim Gama(\alpha, b)$, por lo tanto, $f(\alpha, \lambda) = f(\lambda|\alpha) f(\alpha)$. Así, para la distribución posterior se tiene:
	\begin{equation} \label{eq:8}
		f(\alpha,\lambda|\bar{t}) \propto f(\bar{t}|\alpha, \lambda) f(\alpha, \lambda)
	\end{equation}
	A partir del algoritmo MH usando Kerneles híbridos simule valores de la distribución posterior $f(\alpha, \lambda | \bar{t})$, considerando las siguientes propuestas:
	\begin{itemize}
		\item Propuesta 1:
		\begin{equation} \label{eq:9}
			\lambda_p | \alpha, \bar{t} \sim Gama \left( \alpha+n, b+\sum_{i=1}^{n} t_i^{\alpha} \right) \text{ y dejando $\alpha$ fijo}.
		\end{equation}
		\item Propuesta 2:
		\begin{equation} \label{eq:10}
			\alpha_p | \lambda, \bar{t} \sim Gama \left( n+1, -\log{b} - \log{r_1} + c \right) \text{ con } r_1 = \prod_{i=1}^{n} t_i \text{ y dejando $\lambda$ fijo}.
		\end{equation}
		\item Propuesta 3:
		\begin{equation} \label{eq:11}
			\alpha_p \sim exp(c) \textit{ y } \lambda_p|\alpha_p\sim Gama(\alpha_p,b)
		\end{equation}
		\item Propuesta 4 (RWMH):
		\begin{equation} \label{eq:12}
			\alpha_p = \alpha + \epsilon \text{, con } \epsilon\sim\mathcal{N}(0,\sigma) \text{ y dejando $\lambda$ fijo}.
		\end{equation}
		Simular datos usando $\alpha=\lambda=1$ con $n= 30$. Para la a priori usar $c= 1$ y $b= 1$.
	\end{itemize}
\end{enumerate}

\textcolor{BrickRed}{\it Respuesta:}

En el archivo \textcolor{mediumblue}{ejercicio2\_tarea8.py}, se comienza implementando la función \textit{plot\_chains()}, la cual recibe una cadena de Markov simulada y grafica el histograma de cada variable, además, traza los valores reales de $\alpha$ y $\beta$, que en ejercicio se piden ambos igual a $1$. Gracias a lo general que es la función \textit{METROPOLIS\_HASTINGS\_HYBRID\_KERNELS()} descrita en el ejercicio anterior, se vuelve a usar sin hacer ninguna modificación.

A continuación, se define la función \textit{posterior()} la cual implementa la expresión \eqref{eq:8} y las propuestas:
\begin{itemize}
	\item \textit{prop1\_gen()}  y \textit{prop1\_pdf()}
	\item \textit{prop2\_gen()} y \textit{prop2\_pdf()}
	\item \textit{prop3\_gen()}  y \textit{prop3\_pdf()}
	\item \textit{prop4\_gen()} y \textit{prop4\_pdf()}
\end{itemize}
las cuales implementan las propuestas \eqref{eq:9}, \eqref{eq:10}, \eqref{eq:11} y \eqref{eq:12} así como su densidad de probabilidad respectiva. En estas funciones es importante notar que se respeta cuando se deja fija alguna de las variables ($\alpha$ en la propuesta $1$ y  $\lambda$ en la propuesta $2$ y $4$).

Finalmente, se ejecuta el código principal, en el cual se definen los parámetros dados por el ejercicio, como $\alpha=\lambda=b=c=1$ y $n= 30$. Además, para la propuesta $4$, se usa $\sigma=0.1$ gracias a que se obtuvieron buenos resultados con este valor. El punto $x_0$ inicial se tomó aleatorio en $[0,1]$ por simplicidad y se usaron $10,000$ iteraciones del algoritmo. Además, se le dio igual probabilidad de elección a cada una de las propuestas: $\frac{1}{4}$. Así, se generaron las observaciones $t$ de la distribución Weibull para luego generar las listas de funciones generadoras y sus respectivas densidades para aplicar el algoritmo Metropolis Hastings con Kérneles Híbridos generando los resultados mostrados a continuación:

\begin{figure}[h!]
	\centering
	\includegraphics[width=\textwidth]{IMAGENES/exercise2.pdf}
\end{figure}

Podemos notar que las estimaciones para $\alpha$ y $\lambda$ fueron bastante buenas, ya que las distribuciones tienen sus medias muy cerca a los valores reales $\alpha=\lambda=1$. Este resultado se repitió en varias ocasiones para distintos valores de los parámetros y número de iteraciones, en particular, parecía invariante a la elección de la desviación estándar $\sigma$ de la propuesta $4$.
% ------------------------------------------------------------------------------------
\newpage
\section{Ejercicio 3}
	Investiga y describe muy brevemente los softwares OpenBugs, Nimble, JAGS, DRAM, Rtwalk, Mcee Hammer, PyMCMC.

\vspace{5mm}
{\color{gray} \hrule}
\textcolor{BrickRed}{\it Respuesta:}

\begin{itemize}
	\item \textbf{OpenBugs} (\textit{Bayesian inference Using Gibbs Sampling}) 
\end{itemize}
Es un software diseñado para realizar análisis estadístico bayesiano mediante algoritmos MCMC, particularmente Gibbs Sampling. Es ampliamente utilizado para modelar datos complejos y especificar modelos probabilísticos jerárquicos en su propio lenguaje. OpenBUGS es bien conocido en la comunidad estadística y se integra fácilmente con R a través de paquetes como BRugs. Sin embargo, presenta limitaciones en términos de flexibilidad y rendimiento en problemas de gran escala o alta dimensionalidad.

\begin{itemize}
	\item \textbf{Nimble}
\end{itemize}
Es una plataforma en R que permite construir y personalizar modelos estadísticos, especialmente para MCMC y algoritmos jerárquicos bayesianos. Su lenguaje es similar al de BUGS, pero ofrece la capacidad de optimizar y modificar directamente los algoritmos. Es una herramienta flexible y eficiente, ideal para simulaciones y análisis avanzados, aunque su curva de aprendizaje puede ser pronunciada para principiantes.

\begin{itemize}
	\item \textbf{JAGS} (\textit{Just Another Gibbs Sampler})
\end{itemize}
Es una alternativa modular y extensible a OpenBUGS, diseñada para el muestreo MCMC en modelos bayesianos jerárquicos. Compatible con R mediante el paquete rjags, JAGS soporta una amplia variedad de distribuciones y modelos. Ofrece mayor adaptabilidad y modernidad que OpenBUGS, aunque sigue siendo menos flexible que herramientas como Nimble.

\begin{itemize}
	\item \textbf{DRAM} (\textit{Delayed Rejection Adaptive Metropolis})
\end{itemize}
Es una extensión del algoritmo Metropolis-Hastings que combina la adaptabilidad con retraso en el rechazo de propuestas. Esto permite una exploración más efectiva en el espacio de parámetros, especialmente en distribuciones complejas o con correlaciones fuertes. DRAM reduce el tiempo de convergencia en comparación con Metropolis-Hastings estándar, aunque puede ser más costoso computacionalmente.

\begin{itemize}
	\item \textbf{Rtwalk}
\end{itemize}
Es un algoritmo MCMC basado en “random walk” diseñado para explorar eficientemente espacios de alta dimensión. Garantiza irreducibilidad y está optimizado para problemas bayesianos con restricciones o geometrías complejas. Es particularmente eficiente en problemas con muchas dimensiones, pero su uso es menos extendido, lo que limita el soporte comunitario disponible.

\begin{itemize}
	\item \textbf{Mcee Hammer}
\end{itemize}
Es un sampler MCMC especializado inicialmente en problemas de modelado no lineal en astronomía. Utiliza un enfoque de conjunto de puntos (ensemble sampler), como el algoritmo Affine Invariant, para manejar modelos con muchas dimensiones y correlaciones fuertes. Es fácil de usar, ya que no requiere un ajuste detallado de parámetros, y aunque es popular en astronomía, también puede aplicarse en otros campos científicos.

\begin{itemize}
	\item \textbf{PyMCMC}
\end{itemize}
Es una biblioteca en Python para implementar métodos MCMC personalizados, especialmente en inferencia bayesiana. Integra herramientas del ecosistema de Python como NumPy, SciPy y Matplotlib, y permite adaptar los algoritmos de muestreo a necesidades específicas. Su flexibilidad es ideal para usuarios que desean personalizar sus modelos, aunque tiene menos soporte comunitario en comparación con herramientas más maduras como Stan o PyMC.

\end{document}
